\chapter{\IfLanguageName{dutch}{Stand van zaken}{State of the art}}
\label{ch:stand-van-zaken}

% Tip: Begin elk hoofdstuk met een paragraaf inleiding die beschrijft hoe
% dit hoofdstuk past binnen het geheel van de bachelorproef. Geef in het
% bijzonder aan wat de link is met het vorige en volgende hoofdstuk.

% Pas na deze inleidende paragraaf komt de eerste sectiehoofding.
In een eerste hoofdstuk wordt het begrip artificiële intelligentie besproken. Deze term is heel belangrijk, want het analyseren van reviews is een toepassing hiervan. Ook de geschiedenis wordt hier kort besproken. Daarna wordt machine learning beter uitgelegd. Machine learning is een belangrijk deelveld van artificiële intelligentie. Vervolgens worden deep learning en neurale netwerken besproken. Deze technieken zijn nodig om de reviews te analyseren. 

\section{Artificiële Intelligentie: Introductie en Termen}
\label{sec:artificiëleintelligentieintroductie}

\subsection{Artificiële Intelligentie}
\label{sec:artificiëleintelligentie}
Een van de hoofdbegrippen van deze bachelorproef is artificiële intelligentie (AI). Artificiële intelligentie is een tak van de wetenschap die zich bezighoudt met het onderzoek naar hoe computers of machines het denkvermogen, leervermogen en keuzeproces van de mens proberen na te bootsen. Het hoofddoel van AI is om naar de vaardigheden van de mens te kunnen handelen. \autocite{IBM2021}

AI is echter niet zo’n vergezocht concept als velen denken. AI bevindt zich overal rondom ons, ook in het dagelijkse leven. Enkele voorbeelden: een chatbot van een website waarop iemand een item wilt bestellen, fotoherkenning of aanbevelingen van ons favoriete streamingplatform. \autocite{IBM2021}

Artificiële intelligentie kan opgedeeld worden in cognitieve intelligentie en emotionele intelligentie. Onder cognitieve intelligentie wordt begrepen dat een AI zijn ervaring gebruikt om toekomstige beslissingen te maken. Bij emotionele intelligentie probeert de AI de mens en zijn emoties echt te begrijpen. \autocite{Andreas2018}

\textcite{Kraaijvanger2012} schreef een artikel over de Turing Test. Dit is een begrip dat vaak met artificiële intelligentie geassocieerd wordt. Bij een Turing Test wordt er gekeken of een computer (AI) een persoon kan laten denken dat de computer een mens is. Is dit het geval, dan wordt de computer aanzien als intelligent. Deze Turing Test werd in het leven geroepen in 1950 door Alan Turing. 

\subsection{De geschiedenis van Artificiële Intelligentie}
\label{sec:artificiëleintelligentiegeschiedenis}

Officieel werd er voor het eerst over het onderzoeksgebied artificiële intelligentie gesproken in 1956, op een conferentie in Hanover, New Hampshire. Hier werd de term ‘kunstmatige intelligentie’ voor het eerst in het leven geroepen. Het concept artificiële intelligentie was niet gemakkelijk en de vooruitgang vorderde traag, waardoor in de jaren ‘70 de interesse voor AI snel daalde. Vanaf de jaren ‘80 begon de Britse regering het onderzoek naar AI opnieuw te financieren, waardoor vooruitgang opeens versnelde. \autocite{Anyoha2017}

In het artikel van \textcite{Anyoha2017} wordt nadruk gelegd op het jaar 1997. Dit was een belangrijk jaar voor de wetenschap. Dit was het jaar waarin Deep Blue, een computer van IBM de Russische schaakmeester Garry Kasparov versloeg. AI werd vanaf dan gezien als een ‘hot topic’ en deze overwinning betekende een grote stap vooruit voor artificiële intelligentie. In ditzelfde jaar ontwikkelde Dragon Systems spraakherkenningssoftware die geïmplementeerd kon worden in Windows. 
In 2014 werd de bekende chatbot ‘Eugene Goostman’ gemaakt. Deze chatbot nam deel aan een wedstrijd, waarbij 33\% van de jury de robot als een mens aanzag. Velen zijn daarom van mening dat de chatbot de hierboven besproken Turing Test doorstaan heeft. \autocite{Anyoha2017}
 
Artificiële intelligentie blijft tot op vandaag de dag een veelbesproken onderwerp en vooruitgang op dit gebied gaat aan een razend tempo. 

\subsection{Machine Learning}
\label{sec:machinelearning}

\subsubsection{Supervised Learning}
\label{sec:supervisedlearning}

\subsubsection{Unsupervised Learning}
\label{sec:unsupervisedlearning}

\subsubsection{Reinforcement Learning}
\label{sec:reinforcementlearning}

\subsection{Deep Learning}
\label{sec:deeplearning}

\subsection{Neurale Netwerken}
\label{sec:neurallnetworks}



