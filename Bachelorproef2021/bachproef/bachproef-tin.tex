%===============================================================================
% LaTeX sjabloon voor de bachelorproef toegepaste informatica aan HOGENT
% Meer info op https://github.com/HoGentTIN/bachproef-latex-sjabloon
%===============================================================================

\documentclass{bachproef-tin}

\usepackage{hogent-thesis-titlepage} % Titelpagina conform aan HOGENT huisstijl

%%---------- Documenteigenschappen ---------------------------------------------
% TODO: Vul dit aan met je eigen info:

% De titel van het rapport/bachelorproef
\title{In welke mate kunnen AI’s getraind worden om menselijke
    gevoelswaarden te herkennen in teksten?}

% Je eigen naam
\author{Eline Moerman}

% De naam van je promotor (lector van de opleiding)
\promotor{Leen Vuyge}

% De naam van je co-promotor. Als je promotor ook je opdrachtgever is en je
% dus ook inhoudelijk begeleidt (en enkel dan!), mag je dit leeg laten.
\copromotor{Robin Menschaert, Lothar Verledens}

% Indien je bachelorproef in opdracht van/in samenwerking met een bedrijf of
% externe organisatie geschreven is, geef je hier de naam. Zoniet laat je dit
% zoals het is.
\instelling{Inetum-Realdolmen}

% Academiejaar
\academiejaar{2020-2021}

% Examenperiode
%  - 1e semester = 1e examenperiode => 1
%  - 2e semester = 2e examenperiode => 2
%  - tweede zit  = 3e examenperiode => 3
\examenperiode{2}
\makeglossaries

\newglossaryentry{AI}
{
    name=Artificiële Intelligentie,
    description={Een tak van de wetenschap die zich bezighoudt met het onderzoek naar hoe computers of machines het denkvermogen, leervermogen en keuzeproces van de mens proberen na te bootsen.}
}

\newglossaryentry{MachineLearning}
{
    name=Machine Learning,
    description={Een deelveld van artificiële intelligentie waarbij de machine zichzelf programmeert om een specifieke taak zo goed mogelijk uit te voeren.}
}

\newglossaryentry{TuringTest}
{
    name=Turing Test,
    description={Een test om te beoordelen of een computer aanzien kan worden als intelligent door een persoon te laten denken dat de computer een mens is.}
}
\newglossaryentry{SupervisedLearning}
{
    name=Supervised Learning,
    description={De machine krijgt gelabelde gegevens om te analyseren.}
}
\newglossaryentry{UnsupervisedLearning}
{
    name=Unsupervised Learning,
    description={De machine krijgt ongelabelde gegevens om te analyseren met als doel om een structuur te ontdekken in deze gegevens.}
}
\newglossaryentry{ReinforcementLearning}
{
    name=Reinforcement Learning,
    description={De machine leert aan de hand van een aantal beloningssignalen.}
}
\newglossaryentry{DeepLearning}
{
    name=Deep Learning,
    description={Een deelveld van machine learning waarbij de machine zichzelf een taak aanleert en de gegevens steeds blijft analyseren om de beste nauwkeurigheid te behalen.}
}
\newglossaryentry{NeuraalNetwerk}
{
    name=Neuraal Netwerk,
    description={Een netwerk dat gebaseerd is op de hersenstructuur. Het bestaat uit neuronen.}
}

%===============================================================================
% Inhoud document
%===============================================================================

\begin{document}

%---------- Taalselectie -------------------------------------------------------
% Als je je bachelorproef in het Engels schrijft, haal dan onderstaande regel
% uit commentaar. Let op: de tekst op de voorkaft blijft in het Nederlands, en
% dat is ook de bedoeling!

%\selectlanguage{english}

%---------- Titelblad ----------------------------------------------------------
\inserttitlepage

%---------- Samenvatting, voorwoord --------------------------------------------
\usechapterimagefalse
%%=============================================================================
%% Voorwoord
%%=============================================================================

\chapter*{\IfLanguageName{dutch}{Woord vooraf}{Preface}}
\label{ch:voorwoord}

Ik heb altijd al een passie gehad voor talen. Daarom ben ik in 2015 aan de opleiding Bedrijfsvertaler-tolk begonnen aan de Hogeschool Gent. Na dit diploma behaald te \\hebben, voelde ik dat er nog iets miste. Ik had namelijk ook een grote interesse in IT, webapplicaties en apps. Daarom ving ik in 2018 de studierichting Toegepaste Informatica aan, eveneens aan de Hogeschool Gent.

Om mijn twee grote passies te combineren, kwam ik al snel met het idee op de proppen om met NLP en AI te werken. Het is immers heel interessant om te zien hoe twee volledig verschillende richtingen toch samen kunnen komen op bepaalde raakvlakken.

Ik kwam het eerst in aanraking met Artificiële Intelligentie in het derde jaar tijdens de lessen Artificiële Intelligentie en Databanken III. Ik vond dit direct een intrigerend \\onderwerp en heb daarop ook mijn keuze voor het onderwerp van deze bachelorproef gebaseerd. 

Deze bachelorproef is de kers op de taart van mijn laatste jaar Toegepaste Informatica en ben dan ook enorm trots om dit te hebben verwezenlijkt. Maar ik zou dit zeker niet alleen gekunnen hebben.

Allereerst wil ik mijn promotor, Leen Vuyge bedanken om me bij te staan met goede raad en om steeds mijn taalfoutjes uit de bachelorproef te halen. Het was een enorm voorrecht om met u samen te werken en samen dit concept uit te denken.

Ook wil ik mijn twee co-promotors Robin Menschaert en Lothar Verledens bedanken voor de expertise en suggesties. Dankzij hun durfde ik het aan om ook zelf een proof-of-concept te proberen schrijven.

Ten slotte wil ik mijn revisoren bedanken. Bedankt Joppe Eggermont, Joachim 'Jote' Demyttenaere, Isabelle Schelstraete en Lieven Moerman om mijn bachelorproef na te lezen en alle foutjes eruit te halen.

Verder wens ik u veel leesplezier toe en hoop ik dat u een beter inzicht krijgt in NLP en Opinion Mining.

Eline Moerman \\
28/05/2021, Gent

%%=============================================================================
%% Samenvatting
%%=============================================================================

% TODO: De "abstract" of samenvatting is een kernachtige (~ 1 blz. voor een
% thesis) synthese van het document.
%
% Deze aspecten moeten zeker aan bod komen:
% - Context: waarom is dit werk belangrijk?
% - Nood: waarom moest dit onderzocht worden?
% - Taak: wat heb je precies gedaan?
% - Object: wat staat in dit document geschreven?
% - Resultaat: wat was het resultaat?
% - Conclusie: wat is/zijn de belangrijkste conclusie(s)?
% - Perspectief: blijven er nog vragen open die in de toekomst nog kunnen
%    onderzocht worden? Wat is een mogelijk vervolg voor jouw onderzoek?

\IfLanguageName{english}{%
\selectlanguage{dutch}
\chapter*{Samenvatting}
\lipsum[1-4]
\selectlanguage{english}
}{}

%%---------- Samenvatting -----------------------------------------------------
% De samenvatting in de hoofdtaal van het document

\chapter*{\IfLanguageName{dutch}{Samenvatting}{Abstract}}

Artificiële Intelligentie is een concept dat de laatste jaren meer en meer in opmars is. Natural Language Processing, het deelveld van AI dat zich met taal bezighoudt, kent steeds meer vooruitgang. Denk maar aan Alexa, Siri, Google Translate of spraakgestuurde huizen. Ook bedrijven kijken vaker in de richting van NLP om de reviews te kunnen analyseren. Het is immers handig om de review van een klant als positief, neutraal of negatief in te kunnen delen. Er kan dan gepast gereageerd worden op de negatieve reviews. Dit zorgt op zijn beurt voor een betere verhouding tussen bedrijf en klant.

In deze bachelorproef werd onderzocht in welke mate AI deze menselijke gevoelswaarden al kan analyseren en er wordt in de conclusie beslist of NLP klaar is om in de bedrijfswereld toegepast te worden. Ook werd er onderzocht hoe ver NLP al staat op dit vlak, en werd er zelf een proof of concept geschreven om te beoordelen of het implementeren van een AI vlot kan plaatsvinden. 

Om te onderzoeken hoe ver NLP al staat, werden twee externe tools beoordeeld. Een nauwkeurigheid van 80\% betekende dat de tools goed genoeg waren om in het bedrijfsleven toegepast te worden. Een eerste tool, Azure Text Analytics API scoorde heel goed op een dataset met reviews, maar minder goed op een dataset met tweets. Als enkel de reviews beoordeeld worden, zou deze tool meer dan goed genoeg zijn. Voor kortere zinnen, zoals bij tweets, scoorde deze tool ondermaats. De tweede tool, Google Cloud Platform, scoorde op alle datasets veel beter dan de Azure Text Analytics API. 

Ten slotte werden er in de proof of concept een aantel modellen gebruikt om de reviews en tweets te analyseren. Model 1 leverde zowel voor de reviews als voor de tweets goede resultaten op. Model 2 leverde ook redelijke goede resultaten op, maar dit model is niet aan te raden aangezien er soms overfitting optrad. Model 3 leverde redelijke resultaten op, maar dit model wordt bijgesteld in model 4. Model 4 bleek uiteindelijk het beste model te zijn en haalde de 80\% nauwkeurigheid die deze bachelorproef wou bereiken.

Men kan concluderen dat Natural Language Processing klaar is om in de bedrijfswereld gebruikt te worden. Het is het verstandigst om een tool als Google Cloud Platform te gebruiken om zinnen te analyseren. Deze leverde immers de beste resultaten op.

In de toekomst kan het interessant zijn om te onderzoeken hoe er gepast kan gereageerd worden op een negatieve review. Ten eerste zal het onderwerp van de review moeten bepaald worden. Ten tweede zal de AI dan de juiste tekst moeten formuleren om zo de klant tevreden te houden. Er is ook nog verbetering mogelijk op vlak van nauwkeurigheid. In de toekomst kan er onderzoek gedaan worden naar modellen die een nauwkeurigheid van 90\% kunnen halen.









%---------- Inhoudstafel -------------------------------------------------------
\pagestyle{empty} % Geen hoofding
\tableofcontents  % Voeg de inhoudstafel toe
\cleardoublepage  % Zorg dat volgende hoofstuk op een oneven pagina begint
\pagestyle{fancy} % Zet hoofding opnieuw aan

%---------- Lijst figuren, afkortingen, ... ------------------------------------

% Indien gewenst kan je hier een lijst van figuren/tabellen opgeven. Geef in
% dat geval je figuren/tabellen altijd een korte beschrijving:
%
%  \caption[korte beschrijving]{uitgebreide beschrijving}
%
% De korte beschrijving wordt gebruikt voor deze lijst, de uitgebreide staat bij
% de figuur of tabel zelf.

\listoffigures
\listoftables

% Als je een lijst van afkortingen of termen wil toevoegen, dan hoort die
% hier thuis. Gebruik bijvoorbeeld de ``glossaries'' package.
% https://www.overleaf.com/learn/latex/Glossaries

%---------- Kern ---------------------------------------------------------------

% De eerste hoofdstukken van een bachelorproef zijn meestal een inleiding op
% het onderwerp, literatuurstudie en verantwoording methodologie.
% Aarzel niet om een meer beschrijvende titel aan deze hoofstukken te geven of
% om bijvoorbeeld de inleiding en/of stand van zaken over meerdere hoofdstukken
% te verspreiden!

%%=============================================================================
%% Inleiding
%%=============================================================================

\chapter{\IfLanguageName{dutch}{Inleiding}{Introduction}}
\label{ch:inleiding}

Artificiële intelligentie heeft de afgelopen jaren al heel wat vooruitgang geboekt. Ook op het vlak van vertalingen zien we veel tools opduiken zoals Google Translate en DeepL die een redelijk goede vertaling aan de gebruiker kunnen aanbieden. Vaak volstaan deze vertalingen echter niet omdat Artificiële Intelligentie geen emotionele gevoelswaarden of culturele aspecten, zoals beleefdheidsvormen, kan verstaan of vertalen. De betekenis of zelfs context van veel zinnen gaat zo verloren. Maar zou het niet handig zijn moest een AI gevoelswaarden kunnen afleiden uit een tekst? Denk maar aan bedrijven waarvoor de klanten recensies kunnen schrijven. De AI zou dan aan de hand van detectiesoftware bepaalde gevoelswaarden kunnen herkennen. Het bedrijf in kwestie kan dan inspelen op de gevoelens van elke afzonderlijke klant, wat enorm veel voordelen met zich oplevert.
Een voorbeeld hiervan: een klant laat een boze, ontevreden recensie achter. De AI detecteert dat dit een boze recensie is, en speelt hierop in met een verontschuldigend antwoord en misschien zelfs een compensatie.  
In deze bachelorproef zal dit thema onderzocht worden: kunnen AI' s getraind worden om gevoelswaarden in een tekst te begrijpen en hiernaar correct te handelen? Daarbij worden volgende vragen onderzocht:

\begin{itemize}
    \item Hoe ver staan Natural Language Processing en Sentiment Analysis al?
    \item Kunnen AI's bepaalde gevoelswaarden al begrijpen?
    \item In welke mate kunnen AI's deze gevoelswaarden begrijpen?
    \item Kunnen bedrijven Sentiment Analysis en Opinion Mining gebruiken bij het analyseren van hun reviews?
\end{itemize}


\section{\IfLanguageName{dutch}{Probleemstelling}{Problem Statement}}
\label{sec:probleemstelling}

Veel bedrijven die producten of diensten aanbieden hebben een pagina waarop de klant reviews kan achterlaten. Het is echter niet gemakkelijk om elke review te bekijken, analyseren en hier gepast naar te handelen. Het doel van reviews is om de klant en zijn gevoelens beter te begrijpen. Het zou een enorme meerwaarde voor het bedrijf bieden, mochten de reviews van de klant geanalyseerd kunnen worden door artificiële intelligentie. Zo kan er gepast gereageerd worden op een review of klacht. 

\section{\IfLanguageName{dutch}{Onderzoeksvraag}{Research question}}
\label{sec:onderzoeksvraag}

Om te onderzoeken of artificiële intelligentie sentimentele data kan halen uit reviews en/of klachten, luidt de onderzoeksvraag van deze bachelorproef als volgt: `In welke mate kunnen AI's getraind worden om menselijke gevoelswaarden te herkennen in teksten?'. Deze gevoelswaarden zullen positief, neutraal of negatief blijken te zijn. Ook worden er een paar deelvragen onderzocht om het onderwerp beter te kaderen:

\begin{itemize}
    \item Hoe ver staan Natural Language Processing en Sentiment Analysis al?
    \item Kunnen AI's bepaalde gevoelswaarden al begrijpen?
    \item In welke mate kunnen AI's deze gevoelswaarden begrijpen?
    \item Kunnen bedrijven Sentiment Analysis en Opinion Mining gebruiken bij het analyseren van hun reviews?
\end{itemize}

Op het einde van deze bachelorproef zal duidelijk zijn of Sentiment Analysis kan gebruikt worden door bedrijven om een review te analyseren. Verder wordt er ook onderzocht of dit effectief is. Haalt het bedrijf hier echt voordeel uit? Zijn de voorspellingen voor een review accuraat?


\section{\IfLanguageName{dutch}{Onderzoeksdoelstelling}{Research objective}}
\label{sec:onderzoeksdoelstelling}

Het resultaat dat deze bachelorproef wil bereiken is om een betere verstandshouding tussen het bedrijf en de klant te scheppen. Wanneer reviews geanalyseerd kunnen worden door een machine, kan deze de reviews indelen als positief, neutraal of negatief. Daarna kan er gepast geantwoord worden op de review van de klant. 

Deze bachelorproef kan als 'succesvol' gezien worden wanneer een conclusie bereikt wordt over het punt dat artificiële intelligentie al dan niet klaar is om sentiment analysis toe te passen op reviews. Een succespercentage van 80 procent nauwkeurigheid wordt gezien als succesvol. Er worden twee tools getest om te kijken of NLP klaar is om door het bedrijfsleven toegepast te worden. Verder wordt er ook een eigen versie geschreven om dit percentage te bereiken.

\section{\IfLanguageName{dutch}{Opzet van deze bachelorproef}{Structure of this bachelor thesis}}
\label{sec:opzet-bachelorproef}

% Het is gebruikelijk aan het einde van de inleiding een overzicht te
% geven van de opbouw van de rest van de tekst. Deze sectie bevat al een aanzet
% die je kan aanvullen/aanpassen in functie van je eigen tekst.

De rest van deze bachelorproef is als volgt opgebouwd:

In Hoofdstuk~\ref{ch:stand-van-zaken} wordt een overzicht gegeven van de stand van zaken binnen het onderzoeksdomein, op basis van een literatuurstudie. Hier worden ook enkele belangrijke termen besproken die nodig zijn om mee te kunnen volgen met het NLP-verhaal.

In Hoofdstuk~\ref{ch:methodologie} worden de twee externe tools onderzocht om te beoordelen of deze geschikt zijn om in het bedrijfsleven toegepast te worden.

In Hoofdstuk~\ref{ch:proof-of-concept} worden drie datasets getraind om te kijken wat de nauwkeurigheid is dat hier uit komt en dit te vergelijken met de twee eerder vermelde externe tools.a

In Hoofdstuk~\ref{ch:conclusie} wordt de conclusie gegeven en worden de deelvragen één voor één beantwoord. 

Achteraan deze bachelorproef kan de lezer een woordenlijst vinden. Wanneer hij of zij tijdens het lezen een term niet begrijpt, kan deze gemakkelijk opgezocht worden in de glossary.
\chapter{\IfLanguageName{dutch}{Stand van zaken}{State of the art}}
\label{ch:stand-van-zaken}

% Tip: Begin elk hoofdstuk met een paragraaf inleiding die beschrijft hoe
% dit hoofdstuk past binnen het geheel van de bachelorproef. Geef in het
% bijzonder aan wat de link is met het vorige en volgende hoofdstuk.

% Pas na deze inleidende paragraaf komt de eerste sectiehoofding.
In een eerste hoofdstuk wordt het begrip artificiële intelligentie besproken. Deze term is heel belangrijk, want het analyseren van reviews is een toepassing hiervan. Ook de geschiedenis wordt hier kort besproken. Daarna wordt machine learning beter uitgelegd. Machine learning is een belangrijk deelveld van artificiële intelligentie. Vervolgens worden deep learning en neurale netwerken besproken. Deze technieken zijn nodig om de reviews te analyseren. 

\section{Artificiële Intelligentie: Introductie en Termen}
\label{sec:artificiëleintelligentieintroductie}

\subsection{Artificiële Intelligentie}
\label{sec:artificiëleintelligentie}
Een van de hoofdbegrippen van deze bachelorproef is artificiële intelligentie (AI). Artificiële intelligentie is een tak van de wetenschap die zich bezighoudt met het onderzoek naar hoe computers of machines het denkvermogen, leervermogen en keuzeproces van de mens proberen na te bootsen. Het hoofddoel van AI is om naar de vaardigheden van de mens te kunnen handelen. \autocite{IBM2021}

AI is echter niet zo’n vergezocht concept als velen denken. AI bevindt zich overal rondom ons, ook in het dagelijkse leven. Enkele voorbeelden: een chatbot van een website waarop iemand een item wilt bestellen, fotoherkenning of aanbevelingen van ons favoriete streamingplatform. \autocite{IBM2021}

Artificiële intelligentie kan opgedeeld worden in cognitieve intelligentie en emotionele intelligentie. Onder cognitieve intelligentie wordt begrepen dat een AI zijn ervaring gebruikt om toekomstige beslissingen te maken. Bij emotionele intelligentie probeert de AI de mens en zijn emoties echt te begrijpen. \autocite{Andreas2018}

\textcite{Kraaijvanger2012} schreef een artikel over de Turing Test. Dit is een begrip dat vaak met artificiële intelligentie geassocieerd wordt. Bij een Turing Test wordt er gekeken of een computer (AI) een persoon kan laten denken dat de computer een mens is. Is dit het geval, dan wordt de computer aanzien als intelligent. Deze Turing Test werd in het leven geroepen in 1950 door Alan Turing. 

\subsection{De geschiedenis van Artificiële Intelligentie}
\label{sec:artificiëleintelligentiegeschiedenis}

Officieel werd er voor het eerst over het onderzoeksgebied artificiële intelligentie gesproken in 1956, op een conferentie in Hanover, New Hampshire. Hier werd de term ‘kunstmatige intelligentie’ voor het eerst in het leven geroepen. Het concept artificiële intelligentie was niet gemakkelijk en de vooruitgang vorderde traag, waardoor in de jaren ‘70 de interesse voor AI snel daalde. Vanaf de jaren ‘80 begon de Britse regering het onderzoek naar AI opnieuw te financieren, waardoor vooruitgang opeens versnelde. \autocite{Anyoha2017}

In het artikel van \textcite{Anyoha2017} wordt nadruk gelegd op het jaar 1997. Dit was een belangrijk jaar voor de wetenschap. Dit was het jaar waarin Deep Blue, een computer van IBM de Russische schaakmeester Garry Kasparov versloeg. AI werd vanaf dan gezien als een ‘hot topic’ en deze overwinning betekende een grote stap vooruit voor artificiële intelligentie. In ditzelfde jaar ontwikkelde Dragon Systems spraakherkenningssoftware die geïmplementeerd kon worden in Windows. 
In 2014 werd de bekende chatbot ‘Eugene Goostman’ gemaakt. Deze chatbot nam deel aan een wedstrijd, waarbij 33\% van de jury de robot als een mens aanzag. Velen zijn daarom van mening dat de chatbot de hierboven besproken Turing Test doorstaan heeft. \autocite{Anyoha2017}
 
Artificiële intelligentie blijft tot op vandaag de dag een veelbesproken onderwerp en vooruitgang op dit gebied gaat aan een razend tempo. 

\subsection{Machine Learning}
\label{sec:machinelearning}

\subsubsection{Supervised Learning}
\label{sec:supervisedlearning}

\subsubsection{Unsupervised Learning}
\label{sec:unsupervisedlearning}

\subsubsection{Reinforcement Learning}
\label{sec:reinforcementlearning}

\subsection{Deep Learning}
\label{sec:deeplearning}

\subsection{Neurale Netwerken}
\label{sec:neurallnetworks}




%%=============================================================================
%% Methodologie
%%=============================================================================

\chapter{\IfLanguageName{dutch}{Methodologie}{Methodology}}
\label{ch:methodologie}

%% TODO: Hoe ben je te werk gegaan? Verdeel je onderzoek in grote fasen, en
%% licht in elke fase toe welke stappen je gevolgd hebt. Verantwoord waarom je
%% op deze manier te werk gegaan bent. Je moet kunnen aantonen dat je de best
%% mogelijke manier toegepast hebt om een antwoord te vinden op de
%% onderzoeksvraag.

In dit gedeelte van de bachelorproef wordt onderzocht of AI klaar is om Sentiment Analysis toe te passen in het bedrijfsleven. Er zullen een aantal mogelijkheden worden getest om te zien welke tool het meest nauwkeurig is. Zoals eerder vermeld, zal Sentiment Analysis klaar zijn om te gebruiken in het bedrijfsleven bij een succespercentage van 80\%. Om dit te realiseren, zullen eerst twee externe tools getest worden: Microsoft Azure Text Analytics API en Google Cloud Platform. In hoofdstuk~\ref{ch:proof-of-concept}: Proof Of Concept, zal er zelf getest worden of er een even goed of beter percentage kan bereikt worden door eigen geschreven modellen.

De eerste tool die getest wordt is de Microsoft Azure Text Analytics API. 

\section{Microsoft Azure Text Analytics API}

\subsection{Achtergrond informatie}
\label{achtergrondinformatieazure}
Zoals eerder besproken, biedt Microsoft Azure SaaS-tools aan in de vorm van software. De Text Analytics API biedt dus ook NLP features aan voor text mining, text analysis, sentiment analysis, opinion mining... \autocite{Microsoft2020} Deze API maakt deel uit van de Azure Cognitive Services. Deze biedt een volledig aanbod aan machine learning en AI algoritmes. Om deze services te gebruiken, moet de gebruiker wel een account aanmaken. \autocite{Microsoft2020}

Om deze Microsoft Text Analytics API te implementeren zal er gebruik gemaakt worden van Microsoft Visual Studio. Dit is een IDE (Integrated Development Environment) en wordt gebruikt om applicaties, websites en software te ontwikkelen voor Windows. Voor deze bachelorproef wordt de code in Microsoft Visual Studio geschreven, met name in de programmeertaal C\#. 

\subsection{Aanpak}
\label{aanpakazure}
\textbf{Stap 1}: De Text Analytics Resource

Om te beginnen moet er een Text Analytics Resource gemaakt worden in Azure zoals te zien op figuur \ref{azureresource}. Een Text Analytics Resource is een service waardoor de gebruiker teksten kan analyseren zonder daarvoor de AI te moeten trainen. \autocite{Microsoft2020}

\begin{figure}[!htbp]
    \includegraphics[width=\textwidth]{AzureResource.PNG}
    \caption{\label{azureresource}De Azure Text Analytics Resource \autocite{Microsoft2021}.}
\end{figure}
\FloatBarrier

Eenmaal de service aangemaakt is, genereert Azure een sleutel en een endpoint. Deze zullen gebruikt worden in Visual Studio om de Azure service te kunnen gebruiken. De key en endpoint zijn nodig zodat niemand anders van deze service kan gebruik maken. 

\textbf{Stap 2}: Een project opzetten in Visual Studio en de juiste packages installeren

Wanneer Visual Studio opgestart wordt, vraagt de applicatie om een nieuw project te maken. Hier is de beste keuze een \gls{.NETCore} console applicatie. Dit zorgt ervoor dat er geen overbodige bestanden worden aangemaakt en dat enkel de klasse Program.cs aangemaakt zal worden. In deze file zal al de code geschreven worden om de datasets te kunnen analyseren. \autocite{Microsoft2020}

Daarna moet er een package geïnstalleerd worden. Een \gls{package} is herbruikbare code die al door andere developers geschreven is en die de gebruiker kan downloaden in zijn of haar project. \autocite{Microsoft2018} De \gls{package} die nodig is, is Azure.AI.TextAnalytics, zoals te zien op figuur \ref{azurepackage}.

\begin{figure}[!htbp]
    \includegraphics[width=\textwidth]{AzurePackage.PNG}
    \caption{\label{azurepackage}De Azure TextAnalytics Package \autocite{Microsoft2020}.}
\end{figure}
\FloatBarrier


\textbf{Stap 3}: De data omzetten naar het juiste formaat

Om de data uit de datasets te gebruiken, moet deze toegevoegd worden in Visual Studio. Allereerst wordt er een nieuwe klasse 'Data' aangemaakt. Hier zal alle data opgeslagen worden zodat deze later kan gebruikt worden. Daarna wordt er een nieuw Google Colab document aangemaakt. Een Google Colab document is een tool waar de gebruiker code in Python kan schrijven. Dit gebeurt allemaal online, er moet geen software gedownload worden om deze tool te kunnen gebruiken. 

\textbf{Stap 4}: De juiste code schrijven om zinnen te kunnen analyseren

Met behulp van de Azure Text Analytics documentatie, kan de code in enkele methoden geschreven worden. Alle code wordt geschreven in de klasse Program.cs. 

\subsection{Amazon Dataset}
\label{amazondatasetazure}

\subsubsection{Data omzetten}
\label{amazondatasetomzettenazure}
Wanneer de Amazon dataset van het internet gehaald wordt, krijgt de gebruiker twee \gls{bz2} bestanden. Deze moeten natuurlijk omgezet worden zodat de data in Visual Studio geanalyseerd kan worden. Het is handig om de bestanden op Google Drive op te slaan aangezien men hier heel gemakkelijk vanuit een Google Colab document aankan. In figuur \ref{stap1amazon} wordt dit proces getoond. Om te beginnen, worden de juiste imports toegevoegd om data te kunnen analyseren. In dit bestand wordt er gebruik gemaakt van \gls{pandas}, \gls{numpy} en \gls{bz2}. Daarna wordt er toegang tot Google Drive gemaakt. In een laatste stap worden de bestanden vanop Google Drive uitgelezen.

\begin{figure}[!htbp]
    \includegraphics[width=\textwidth]{Stap1Omzetten.PNG}
    \caption{\label{stap1amazon}De bestanden worden opgehaald uit Google Drive.}
\end{figure}
\FloatBarrier

Daarna (figuur \ref{stap2amazon} wordt de data omgevormd tot bruikbare zinnen. Om te beginnen wordt de data via de functie readlines() omgezet naar een lijst van items waar elke lijn een object vormt. In het tweede blokje code wordt aangegeven hoe een review van de dataset eruitziet. Men kan zien dat er voor elke zin nog een b staat. Deze b staat er door de functie readlines(). Byte objecten beginnen steeds met een b. Deze b moet weggefilterd worden, zodat de data naar tekst omgevormd wordt. 

Eenmaal dit gebeurd is, kan men in het vierde blokje code zien dat de b weg is.

\begin{figure}[!htbp]
    \includegraphics[width=\textwidth]{Stap2Omzetten.PNG}
    \caption{\label{stap2amazon}De data wordt omgezet naar tekst.}
\end{figure}
\FloatBarrier

Echter is dit nog niet genoeg om met deze data verder te werken. Vooraan elke zin staat ook nog een label. Dit label representeert een positieve of negatieve connotatie van de zin. Daarom worden de labels en de tekst opgeplitst in twee datasets, zoals te zien op figuur \ref{stap3amazon}. De eerste dataset heet train\_labels: deze bevat de sentimenten die bij elke zin horen. Label1 wordt omgezet naar 0, label2 wordt omgezet naar 1. 1 representeert een positieve context, 0 een negatieve context. De tweede dataset heeft als naam train\_sentences: deze dataset bevat de eigenlijke zinnen die door het programma in Visual Studio zullen gehaald worden. In codeblokje twee staat een voorbeeld van hoe een zin er nu uitziet, in codeblokje 3 staat een voorbeeld van hoe een score er nu uitziet. 

\begin{figure}[!htbp]
    \includegraphics[width=\textwidth]{Stap3Omzetten.PNG}
    \caption{\label{stap3amazon}De data wordt opgeplitst.}
\end{figure}
\FloatBarrier

Verder zijn er veel url's die gebruikt worden in de zinnen. Deze zijn niet nodig om de connotatie van een zin te analyseren. Via onderstaande methode in figuur \ref{stap4amazon} worden de url's uit de tekst gefilterd.

\begin{figure}[!htbp]
    \includegraphics[width=\textwidth]{Stap4Omzetten.PNG}
    \caption{\label{stap4amazon}De url's worden uit de zinnen gehaald.}
\end{figure}
\FloatBarrier

Om te visualiseren hoe de data er momenteel uitziet, worden de zinnen in een dataframe geplaatst. Daarna worden de eerste 100 items getoond via de methode head(100).

\begin{figure}[!htbp]
    \includegraphics[width=\textwidth]{Stap5Omzetten.PNG}
    \caption{\label{stap5amazon}De data wordt omgezet naar een DataFrame.}
\end{figure}
\FloatBarrier

Zodat de zinnen gemakkelijk in Visual Studio kunnen toegevoegd worden, worden de eerste 500 zinnen met de methode head(500) geprint en wordt deze omgezet in de vorm Sentences.Add(zin). Zo kan dit gemakkelijk en zonder problemen gekopieerd worden naar Visual Studio (zie figuur \ref{stap6amazon}.

\begin{figure}[!htbp]
    \includegraphics[width=\textwidth]{Stap6Omzetten.PNG}
    \caption{\label{stap6amazon}De zinnen worden in een formaat gegoten dat gemakkelijk in Visual Studio kan gebruikt worden.}
\end{figure}
\FloatBarrier

In figuur \ref{stap7amazon} kan men zien dat de zinnen die hiervoor gegenereerd werden, gekopieerd zijn naar de klasse Data. De zinnen worden in een lijst van \gls{string}s gestopt die men 'Sentences' noemt. De scores worden in een lijst van \gls{int}s gestoken, nadat deze omgezet zijn in figuur \ref{stap8amazon}.

\begin{figure}[!htbp]
    \includegraphics[width=\textwidth]{Stap7Omzetten.PNG}
    \caption{\label{stap7amazon}De zinnen worden toegevoegd in de klasse Data.}
\end{figure}
\FloatBarrier

\begin{figure}[!htbp]
    \includegraphics[width=\textwidth]{Stap8Omzetten.PNG}
    \caption{\label{stap8amazon}De scores worden in een formaat geplaatst dat gemakkelijk in Visual Studio kan gebruikt worden.}
\end{figure}
\FloatBarrier


\subsubsection{Visual Studio}
\label{amazondatasetvisualstudioazure}
Nu de data-omzetting gebeurd is, beschikt Visual Studio over een lijst van 500 zinnen met bijhorende score. Een score 0 betekent dat de zin als 'negatief' beschouwd wordt, terwijl een score van 1 een positieve connotatie voorstelt. Om te kijken of de Azure Text Analytics API de zinnen ook werkelijk juist categoriseert, zal er een bepaalde nauwkeurigheid berekend moeten worden. 

Maar om dit te verwezenlijken, moet er eerst een connectie met de Azure Text Analytics resource gemaakt worden. Dit gebeurt aan de hand van de endpoint en de key. Deze worden bovenaan in de klasse Program.cs geplaatst. Alle code zal vanaf nu in deze klasse geschreven worden. 

\begin{figure}[!htbp]
    \includegraphics[width=\textwidth]{AzureKeyCredentials.PNG}
    \caption{\label{azurecredentials}De endpoint en key worden bovenaan de klasse geplaatst.}
\end{figure}
\FloatBarrier

Eenmaal de connectie in orde is, kan het onderzoek aan de slag gaan met de Azure Text Analytics API. Ten eerste zal er getest worden of de Azure resource alle zinnen ook daadwerkelijk herkent als 'Engels'. In sectie \ref{sec:Languagedetection} wordt besproken wat Language Detection juist is. Deze techniek zal via een geschreven methode toegepast worden op alle zinnen. Als de getetecteerde taal gelijk is aan 'English', dan zal de nauwkeurigheid (in het codevoorbeeld: accuracy) verhoogd worden met 1. Als alle zinnen juist gezien worden als Engels, wordt er een nauwkeurigheid van 500/500 bereikt. 

\begin{figure}[!htbp]
    \includegraphics[width=\textwidth]{LanguageDetectionAmazon.PNG}
    \caption{\label{azurelanguagedetectionamazon}Language Detection in Visual Studio.}
\end{figure}
\FloatBarrier

Ten tweede zal er getest worden of de Azure Text Analytics API Sentiment Analysis juist kan toepassen. In sectie \ref{sec:sentimentanalysis} wordt uitvoerig besproken wat Sentiment Analysis juist is. Via onderstaande methode in figuur \ref{azuresentimentanalysisamazon} zal de nauwkeurigheid van de resource getest worden op 500. Bij deze methode is er echter nog wat toelichting nodig. De Azure Text Analytics API kan de zinnen categoriseren als Positief, Neutraal, Negatief en Gemengd. De Amazon dataset bevat enkel de informatie of een zin positief of negatief is. 

Daarom wordt er bij elke zin die juist positief of juist negatief bestempeld wordt, een punt bij de nauwkeurigheid opgeteld. Wanneer de zin als uitkomst Neutraal of Gemengd krijgt, wordt er per stukje tekst gekeken of dit positief of negatief is. Als er meer positieve stukjes zijn dan er negatieve stukjes zijn, wordt deze zin toch bestempeld als positief. Omgekeerd gebeurt hetzelfde voor negatief. 

\begin{figure}[!htbp]
    \includegraphics[width=\textwidth]{SentimentAnalysisAmazon.PNG}
    \caption{\label{azuresentimentanalysisamazon}Sentiment Analysis in Visual Studio.}
\end{figure}
\FloatBarrier

\subsubsection{Resultaten}
\label{amazondatasetresultatenazure}
Ten eerste werd er getest of de Azure Text Analytics API alle geselecteerde zinnen uit de dataset categoriseert als 'Engels'. In figuur \ref{azurelanguagedetectionamazonresults} ziet men dat de nauwkeurigheid 499/500 of 99.8\% is. Er werd 1 zin als 'Spaans' gecategoriseerd. Dit komt doordat deze dataset effectief een Spaanse zin bevatte. De nauwkeurigheid wordt daarom aangepast naar 500/500 of 100\%.

\begin{figure}[!htbp]
    \includegraphics[width=\textwidth]{LanguageDetectionAmazonResult.PNG}
    \caption{\label{azurelanguagedetectionamazonresults}Language Detection voor de Amazon dataset in Visual Studio: Resultaten.}
\end{figure}
\FloatBarrier 

Ten tweede werd er getest of de Azure resource de zinnen uit de dataset juist kan categoriseren als positief of negatief. In figuur \ref{azuresentimentanalysisamazonresults} kunnen de resultaten hiervan gevonden worden. De Azure Text Analytics API heeft 421/500 zinnen juist toegekend. Omgerekend is dit 84.2\%. In sectie \ref{sec:onderzoeksdoelstelling} wordt er bepaald dat als een AI 80\% van de resultaten juist classificeert, dat deze als 'succesvol' kan gezien worden en dus bij bedrijven kan gebruikt worden. Voor deze dataset is dit dus zeker het geval, maar om te bepalen of dit betrouwbaar is, wordt dit ook getest op een tweede dataset, de Twitter Airlines dataset. 

\begin{figure}[!htbp]
    \includegraphics[width=\textwidth]{SentimentAnalysisAmazonResult.PNG}
    \caption{\label{azuresentimentanalysisamazonresults}Sentiment Analysis voor de Amazon dataset in Visual Studio: Resultaten.}
\end{figure}
\FloatBarrier 

\subsection{Twitter Airlines Dataset}
\label{twitterdatasetazure}

Ook bij de Twitter Airlines dataset worden dezelfde stappen gevolgd zoals eerder besproken in sectie \ref{aanpakazure} Aanpak. Ten eerste zal de data omgezet worden naar het juiste formaat, ten tweede zullen er enkele methodes uitgevoerd worden met deze data en ten slotte zullen de resultaten besproken worden. 

\subsubsection{Data omzetten}
\label{twitterdatasetomzettenazure}
Ook voor deze dataset moet de data omgevormd worden. Deze keer begint men met een csv bestand. een \gls{csv}-bestand bevat 'comma separated values'. Met andere woorden wordt de data gescheiden door een komma. Om te beginnen worden de juiste imports zoals \gls{numpy}, \gls{pandas} en \gls{re} geïmporteerd. Daarna moet er opnieuw verbinding gemaakt worden met Google Drive zodat het juiste csv bestand opgehaald kan worden. 
\begin{figure}[!htbp]
    \includegraphics[width=\textwidth]{Stap1Twitter.PNG}
    \caption{\label{azurestap1twitter}Twitter Airline dataset: Imports afhandelen en connectie met Google Drive maken.}
\end{figure}
\FloatBarrier 

Daarna moeten de onnodige kolommen verwijderd worden, en dit zijn er heel wat. Enkel de scores en de zinnen zelf zijn nodig.
\begin{figure}[!htbp]
    \includegraphics[width=\textwidth]{Stap2Twitter.PNG}
    \caption{\label{azurestap2twitter}Twitter Airline dataset: De onnodige kolommen verwijderen.}
\end{figure}
\FloatBarrier 

Hierna wordt de data opgeplitst. De zinnen worden in een variabele 'zinnen' geplaatst, terwijl de scores in een variabele 'sentiment' geplaatst worden. 
\begin{figure}[!htbp]
    \includegraphics[width=\textwidth]{Stap3Twitter.PNG}
    \caption{\label{azurestap3twitter}Twitter Airline dataset: de score en de zinnen opsplitsen.}
\end{figure}
\FloatBarrier
Ten slotte worden de zinnen, zoals bij de Amazon dataset, in een nieuw formaat afgeprint zodat ze gemakkelijk te kopiëren zijn naar Visual Studio. 
\begin{figure}[!htbp]
    \includegraphics[width=\textwidth]{Stap4Twitter.PNG}
    \caption{\label{azurestap4twitter}Twitter Airline dataset: De zinnen en scores omzetten zodat ze in Visual Studio gebruikt kunnen worden.}
\end{figure}
\FloatBarrier 

\subsubsection{Visual Studio}
\label{twitterdatasetvisualstudioazure}
Voor deze dataset maakt men gebruik van dezelfde endpoint en key als bij de vorige dataset, zoals te zien op figuur \ref{azurecredentials}. Aan de methode voor gebruik te maken van Language Detection moet er in principe bijna niets aangepast worden. Enkel de dataset verandert. De methode die hier toegepast wordt, is ook te zien in figuur \ref{azurelanguagedetectionamazon}.

Ten tweede wordt ook hier getest of de Azure Text Analytics API de zinnen uit deze dataset kan categoriseren. Hiervoor moet de methode enigzins aangepast worden aangezien de Twitter dataset wel neutrale connotatie herkent. In deze dataset worden de zinnen gezien als positief, negatief of neutraal. 

\begin{figure}[!htbp]
    \includegraphics[width=\textwidth]{SentimentAnalysisTwitter.PNG}
    \caption{\label{azuresentimentanalysistwitter}Language Detection voor de Twitter dataset in Visual Studio: Resultaten.}
\end{figure}
\FloatBarrier 


\subsubsection{Resultaten}
\label{twitterdatasetresultatenazure}
Om te beginnen werd hetzelfde als bij de Amazon dataset getest, namelijk of alle zinnen als 'Engels' herkend worden. 
Hier haalt de Azure Text Analytics API een score van 499/500, omgerekend is dit 99.8\%. Één zin werd echter herkend als 'Frans'. Hoe kan dit? Deze zin had veel speciale tekens. 

\begin{figure}[!htbp]
    \includegraphics[width=\textwidth]{AccuracyTwitterDatasetAzure.PNG}
    \caption{\label{azurelanguagedetectiontwitterresults}Language Detection voor de Twitter dataset in Visual Studio: Resultaten.}
\end{figure}
\FloatBarrier 

Ten slotte werd er ook getest of Azure de zinnen juist herkent als positief, neutraal of negatief. In figuur \ref{azuresentimentanalysistwitterresults} kunnen de resultaten hiervan terug gevonden worden. Er werd een score van 334/500 behaald door de Azure Text Analytics API. Omgerekend is dit 66.8\%. Dit is minder dan bij de Amazon dataset. Dit kan verklaard worden doordat 'tweets' korter zijn dan reviews, maar ook meer hashtags en apestaartjes bevatten. Verder worden er meer emoji's en tussentaal gebruikt bij tweets dan bij reviews.

\begin{figure}[!htbp]
    \includegraphics[width=\textwidth]{SentimentAnalysisTwitterResult.PNG}
    \caption{\label{azuresentimentanalysistwitterresults}Sentiment Analysis voor de Twitter dataset in Visual Studio: Resultaten.}
\end{figure}
\FloatBarrier 

\subsection{IMDB Dataset}
\label{imdbdatasetazure}
Ten slotte wordt ook de IMDB dataset in deze tool getest. 

\subsubsection{Data omzetten}
\label{imdbdatasetomzettenazure}
Na het downloaden van de IMDB dataset, is dit in een \gls{csv} formaat. Deze data moet wel nog wat omgevormd worden om gebruikt te kunnen worden. Allereerst worden de juiste imports gedeclareerd, wordt er opnieuw een connectie met Google Drive gemaakt en wordt het bestand ingelezen. 

\begin{figure}[!htbp]
    \includegraphics[width=\textwidth]{Stap1IMDB.PNG}
    \caption{\label{stap1imdb}IMDB dataset: Imports en connectie met Google Drive.}
\end{figure}
\FloatBarrier 

De IMDB dataset bevat maar twee kolommen, de review en de score. Er moeten dus geen kolommen meer verwijderd worden. De data van de kolom sentiment is momenteel 'positive' of 'negative'. Deze worden ten eerste omgezet naar een 1 of een 0. Daarna wordt de functie clean\_html toegepast, om alle html tags, zoals <p></p> te verwijderen. Een voorbeeld van hoe de data er nu uitziet, is te vinden in figuur \ref{stap2imdb}.

\begin{figure}[!htbp]
    \includegraphics[width=\textwidth]{Stap2IMDB.PNG}
    \caption{\label{stap2imdb}IMDB dataset: Omzetten sentiment en html tags verwijderen.}
\end{figure}
\FloatBarrier 

Hierna worden alle reviews die meer dan 3000 karakters hebben verwijderd uit de dataset. Dit wordt gedaan zodat Visual Studio niet té lang moet runnen. 

\begin{figure}[!htbp]
    \includegraphics[width=\textwidth]{Stap3IMDB.PNG}
    \caption{\label{stap3imdb}IMDB dataset: Reviews met meer dan 3000 karakters verwijderen.}
\end{figure}
\FloatBarrier

Hierna worden de reviews opnieuw in een nieuw formaat afgeprint om gemakkelijk om te zetten naar Visual Studio.

\begin{figure}[!htbp]
    \includegraphics[width=\textwidth]{Stap4IMDB.PNG}
    \caption{\label{stap4imdb}IMDB dataset: Omzetten naar ander formaat.}
\end{figure}
\FloatBarrier

\begin{figure}[!htbp]
    \includegraphics[width=\textwidth]{Stap5IMDB.PNG}
    \caption{\label{stap5imdb}IMDB dataset: Omzetten naar ander formaat.}
\end{figure}
\FloatBarrier



\subsubsection{Visual Studio}
\label{imdbdatasetvisualstudioazure}
Om de taal van de reviews te herkennen, wordt dezelfde methode gebruikt als bij de Twitter en Amazon datasets. Deze kan gevonden worden in figuur \ref{azurelanguagedetectionamazon}.

Er werd ook getest of de Azure Text Analytics API de zinnen correct kan categoriseren. De methode die hier gebruikt werd is dezelfde als in figuur \ref{azuresentimentanalysisamazon}. Als de review juist bestempeld wordt als positief of negatief, wordt een punt bij de nauwkeurigheid opgeteld. Wanneer de zin neutraal of gemengd is, wordt de zin in stukjes geanalyseerd en wordt er bepaald of de zin toch negatiever of positiever is. Als er meer positieve stukjes zijn dan er negatieve stukjes zijn, wordt deze zin toch bestempeld als positief. 

\subsubsection{Resultaten}
\label{imdbdatasetresultatenazure}
Werden alle zinnen als Engels herkend? De Azure Text Analytics API behaalde een score van 500/500 voor deze dataset. Omgerekend is dit 100\%. Alle zinnen werden correct herkend.

\begin{figure}[!htbp]
    \includegraphics[width=\textwidth]{AccuracyIMDBDatasetAzure.PNG}
    \caption{\label{azurelanguagedetectionimdbresults}Language Detection voor de IMDB dataset in Visual Studio: Resultaten.}
\end{figure}
\FloatBarrier 

Ten tweede werd er getest of het categoriseren van de reviews uit de dataset juist verloopt. De Azure Text Analytics API kan deze reviews classificeren als positief of negatief. Er werden 376/500 zinnen juist toegekend. Omgerekend is dit 75,5\%. Dit is een matig resultaat, maar zeker niet slecht. 
\begin{figure}[!htbp]
    \includegraphics[width=\textwidth]{SentimentAnalysisIMDBResult.PNG}
    \caption{\label{azuresentimentanalysisimdbresults}Sentiment Analysis voor de IMDB dataset in Visual Studio: Resultaten.}
\end{figure}
\FloatBarrier 



\subsection{Conclusie Azure Text Analytics API}
\label{conclusieAzure}
De Azure Analytics API is heel goed voor Language Detection. Bij de Amazon en de Twitter datasets behaalde de tool een score van 99.8\%. Bij de IMDB dataset behaalde de tool een score van 100\%. Voor Sentiment Analysis waren de resultaten verschillend bij de datasets. De Amazon dataset had een nauwkeurigheid van 84.2\%, terwijl de Twitter dataset een score van 66.8\% behaalde en de IMDB dataset een score van 75.5\% behaalde. Dit is een enorm groot verschil en kan verklaard worden doordat tweets ten eerste meer tussentaal, emoji's, hashtags en apenstaartjes bevatten dan reviews. Ten tweede speelt de lengte van de zinnen ook een belangrijke rol. Reviews zijn meestal langer dan tweets. De Azure Analytics API behaalt zo een gemiddelde score van 75.5\%. Deze bachelorproef onderzoekt echter of Sentiment Analysis specifiek reviews kan analyseren. Als dus het gemiddelde van de Amazon en de IMDB dataset genomen wordt, krijgt men een mooier getal. Het gemiddelde is dan 79.85\%. Dit is ongeveer 80\% en daarom wordt deze tool bestempeld als 'geschikt' voor gebruik in het bedrijfsleven. 

\section{Google Cloud Platform}

Een tweede tool die getest wordt, is Google Cloud Platform. Ook hier zal de nauwkeurigheid van de tool berekend worden.

\subsection{Achtergrond informatie}
\label{achtergrondinformatiegooglecloudplatform}
Google Cloud Platform of GCP is een platform dat uiteraard aangeboden wordt door Google. Google Cloud Platform voorziet tools voor data-analyse, dataopslag en machine learning. \autocite{Bigelow2017}

Verder is Google Cloud Platform onderdeel van het bekende Google Cloud. Sinds 2011 biedt Google deze clouddienst aan, waardoor de gewone gebruiker gebruik kan maken van de producten die op dezelfde infrastructuur als Google draaien. \autocite{Bigelow2017}

Deze tool zal in de puntjes hieronder getest worden. 

\subsection{Aanpak}
\label{aanpakgoogleplatform}

De aanpak die hier toegepast wordt, kan ook teruggevonden worden in de developer's guide van Google. \autocite{Codelabs2021}

\textbf{Stap 1}: Setup

Om Google Cloud Platform te gebruiker, is er een account nodig. Daarom is de eerste stap een account aanmaken. Het enige dat de gebruiker moet doen, is een Visakaart voorzien. Hier wordt geen geld afgehaald, dit is enkel om te bewijzen dat de gebruiker geen robot is. 

Eenmaal het account aangemaakt is, komt men terecht op de homepagina van het platform. Het volgende dat nodig is, is een nieuw project. De verdere setup gebeurt via de Cloud Shell. Dit is een console die gebruikt kan worden binnen het Google Cloud Platform. 

Met behulp van de \gls{Cloudshell} wordt de Natural Language API aan het project toegevoegd. 

\begin{figure}[!htbp]
    \includegraphics[width=\textwidth]{GoogleAccount.PNG}
    \caption{\label{googleaccount}De details van het nieuwe project.}
\end{figure}
\FloatBarrier 

\textbf{Stap 2}: Authenticatie

Om het platform te gebruiken, is er natuurlijk authenticatie nodig. Hiervoor moet er een 'key' gegenereerd worden. Dit gebeurt opnieuw via de \gls{Cloudshell} console. De commando's die hiervoor gebruikt worden, zijn te vinden in figuur \ref{setupgoogleplatform}.

\begin{figure}[!htbp]
    \includegraphics[width=\textwidth]{SetupGooglePlatform.PNG}
    \caption{\label{setupgoogleplatform}Een key genereren voor authenticatie.}
\end{figure}
\FloatBarrier 

\textbf{Stap 3}: Een C\# console applicatie opzetten

De laatste stap omvat het opzetten van een console applicatie. Via het commando 

dotnet new console -n \{Projectnaam\} 

wordt de applicatie aangemaakt. Hierna moet de Google Cloud Nuget \gls{package} toegevoegd worden. Dit gebeurt via het commando

dotnet add package Google.Cloud.Language.V

Nadat alles opgezet is, kan de vereiste code geschreven worden.


\subsection{Amazon Dataset}
\label{amazongoogleplatform}

De data werd op dezelfde manier omgezet als bij het testen van de Azure Text Analytics API. Deze omzetting kan gevonden worden van figuur \ref{stap1amazon} tot figuur \ref{stap8amazon}.

\subsubsection{Code}
\label{amazoncodegoogleplatform}
De code wordt opnieuw geschreven in de klasse Program.cs. Google Cloud Platform werkt met een score die aan elke zin toegekend wordt. Een score tussen -1 en -0.25 betekent dat de tekst negatief is, een score tussen -0.25 en 0.25 betekent een neutrale score en een score tussen 0.25 en 1.0 staat voor een positieve connotatie. Aangezien de Amazon dataset enkel voorbeelden bevat die negatief of positief zijn, zijn de scores verdeeld als volgt: tussen -1 en 0 betekent negatief en tussen 0 en 1 betekent positief. In figuur \ref{codeamazon} kan de geschreven code teruggevonden worden. Eenmaal alle data geanalyseerd is, komt er een nauwkeurigheid uit. 

\begin{figure}[!htbp]
    \includegraphics[width=\textwidth]{CodeAmazon.PNG}
    \caption{\label{codeamazon}Sentiment Analysis voor de Amazon dataset in Google Cloud Platform.}
\end{figure}
\FloatBarrier 

\subsubsection{Resultaten}
\label{amazonresultatengoogleplatform}

Er werd getest of Google Cloud Platform de zinnen uit de Amazon dataset juist kan categoriseren als positief of negatief. De resultaten hiervoor kunnen in figuur\ref{accurracyAmazon} teruggevonden worden. Na het analyseren van de zinnen, behaalde het platform een score van 465/500. Omgerekend is dit 93.0\%. Zoals eerder besproken, wordt een score van 80\% of hoger gezien als succesvol. Daarom kan geconcludeerd worden dat Google Cloud Platform uitstekende resultaten biedt. Om meer zekerheid aan de lezer te bieden dat Google Cloud Platform een goede keuze is, wordt dit ook met de Twitter Airlines dataset getest. 
\begin{figure}[!htbp]
    \includegraphics[width=\textwidth]{AccuracyAmazonGoogle.PNG}
    \caption{\label{accuracyAmazon}Google Cloud Platform resultaten voor de Amazon dataset.}
\end{figure}
\FloatBarrier 

\subsection{Twitter Airlines Dataset}
\label{twittergoogleplatform}

De data werd op dezelfde manier omgezet als bij het testen van de Azure Text Analytics API. Deze omzetting kan gevonden worden van figuur \ref{azurestap1twitter} tot figuur \ref{azurestap4twitter}.

\subsubsection{Code}
\label{twittercodegoogleplatform}
Aangezien de Twitte Airlines dataset wel data over de drie categorieën positief, neutraal en negatief bevat, kan Google Cloud Platform hier nauwkeuriger toegepast worden. Zinnen met een score boven de 0.25 krijgen een lineSentiment met waarde 'positief', zinnen met een score lager dan -0.25 krijgen een lineSentiment met waarde 'negatief' en zinnen met een score tussen de -0.25 en 0.25 krijgen 'neutraal'. Eenmaal alle zinnen door de code gegaan zijn, wordt er een nauwkeurigheid berekend op 500. De code kan teruggevonden worden in figuur \ref{codetwitter}.

\begin{figure}[!htbp]
    \includegraphics[width=\textwidth]{CodeTwitter.PNG}
    \caption{\label{codetwitter}Sentiment Analysis voor de Twitter dataset in Google Cloud Platform.}
\end{figure}
\FloatBarrier 

\subsubsection{Resultaten}
\label{twitterresultatengoogleplatform}
Er werd getest of Google Cloud Platform de zinnen uit de Twitter Airlines dataset ook juist kan categoriseren als positief, neutraal of negatief. De resultaten hiervoor kunnen in figuur \ref{accuracytwitter} teruggevonden worden. Google Cloud Platform behaalde hier een score van 365/500. Omgerekend is dit 73.0\%. Zoals eerder besproken, wordt een score van 80\% of hoger gezien als succesvol. Daarom kan geconcludeerd worden dat Google Cloud Platform redelijk goede resultaten biedt. Er werd eerder geconcludeerd dat de Twitter Airlines dataset steeds minder goede resultaten zal behalen omdat de tekst korter is, en de zinnen in een ander formaat staan. 

\begin{figure}[!htbp]
    \includegraphics[width=\textwidth]{AccuracyTwitterGoogle.PNG}
    \caption{\label{accuracytwitter}Google Cloud Platform resultaten voor de Twitter dataset.}
\end{figure}
\FloatBarrier 

\subsection{IMDB Dataset}
\label{imdbgoogleplatform}
De data werd op dezelfde manier omgezet als bij het testen van de Azure Text Analytics API. De omzetting kan gevonden worden van figuur \ref{stap1imdb} tot figuur \ref{stap5imdb}. Opnieuw worden zinnen met een score boven de 0 gecategoriseerd als 'positief', en zinnen met een score onder de 0, worden als 'negatief' beoordeeld. Er wordt opnieuw een nauwkeurigheid op 500 berekend.

De code die gebruikt wordt, is dezelfde als in figuur \ref{codeamazon}.


\subsubsection{Resultaten}
\label{imdbresultatengoogleplatform}
Na het testen van Google Cloud Platform op de IMDB Movies Dataset, komen er hier resultaten op 500 uit. De reviews konden gezien worden als positief of negatief. Google Cloud Platform behaalde voor deze dataset een score van 446/500. Omgerekend is dit 89.2\%.

\begin{figure}[!htbp]
    \includegraphics[width=\textwidth]{AccuracyIMDBGoogle.PNG}
    \caption{\label{accuracyimdb}Google Cloud Platform resultaten voor de IMDB dataset.}
\end{figure}
\FloatBarrier 
\subsection{Conclusie Google Cloud Platform}
\label{conclusieGoogleCloudPlatform}

Google Cloud Platform is goed voor Sentiment Analysis. De resultaten verschilden echter enorm tussen de datasets. De Amazon dataset had een nauwkeurigheid van 93.0\%, terwijl de Twitter dataset een score van 73.0\% behaalde en de IMDB dataset een score van 89.2\% behaalde. Dit verschil kan verklaard worden doordat tweets meer tussentaal, emoji's, apenstaartjes en andere speciale tekens bevatten. Verder is de lengte van de zinnen in de Twitter dataset veel korter. Google Cloud Platform behaalt zo een gemiddelde score van 85.07\%, dit ligt mooi boven het gewenste 80\% succespercentage. Zonder de Twitter dataset, behaalt Google Cloud Platform een percentage van 91.1\%.




% Voeg hier je eigen hoofdstukken toe die de ``corpus'' van je bachelorproef
% vormen. De structuur en titels hangen af van je eigen onderzoek. Je kan bv.
% elke fase in je onderzoek in een apart hoofdstuk bespreken.

%\input{...}
%\input{...}
%...

%%=============================================================================
%% Conclusie
%%=============================================================================

\chapter{Conclusie}
\label{ch:conclusie}

% TODO: Trek een duidelijke conclusie, in de vorm van een antwoord op de
% onderzoeksvra(a)g(en). Wat was jouw bijdrage aan het onderzoeksdomein en
% hoe biedt dit meerwaarde aan het vakgebied/doelgroep? 
% Reflecteer kritisch over het resultaat. In Engelse teksten wordt deze sectie
% ``Discussion'' genoemd. Had je deze uitkomst verwacht? Zijn er zaken die nog
% niet duidelijk zijn?
% Heeft het onderzoek geleid tot nieuwe vragen die uitnodigen tot verder 
%onderzoek?

De onderzoeksvraag van deze bachelorproef luidde als volgt: `In welke mate kunnen AI's getraind worden om menselijke gevoelswaarden te herkennen in teksten?'. Deze vraag werd beantwoord door ten eerste twee externe tools zoals de Azure Text Analytics API en Google Cloud Platform te onderzoeken in hoofdstuk \ref{ch:methodologie}. Ten tweede werd ook een proof of concept gemaakt om op deze vraag te kunnen beantwoorden in hoofdstuk \ref{ch:proof-of-concept}. Deze bachelorproef onderzocht of AI sentimentele data kan halen uit reviews. Daarom werden er drie datasets gebruikt. Twee van deze dataset bevatten reviews en/of klachten die van de website zelf gehaald werden. Eén dataset bevatte tweets met reviews over vliegtuigmaatschappijen. Dit werd gedaan om variatie te creëren en zo een beter beeld te krijgen van de resultaten.

In welke mate kunnen AI's getraind worden om menselijke gevoelswaarden te herkennen in teksten? Natural Language Processing en Sentiment Analysis staan al heel ver. Er zijn verschillende tools en modellen die aan het vooraf bepaalde 80\% succespercentage voldoen. Google Cloud Platform bleek echter de beste resultaten te behalen.

Er waren ook een aantal deelvragen gedefinieerd. Deze zullen zo goed mogelijk beantwoord worden.

\section{Hoe ver staan Natural Language Processing en Sentiment Analysis al?}
\label{sec:hoever}
Op deze vraag kan duidelijk geantwoord worden dat NLP en Sentiment Analysis al heel ver staan. Zowel voor de externe tools, als voor de modellen werden er redelijk goede resultaten behaald op de drie datasets.
Voor alle modellen en externe tools werd ten minste een score van 75\% behaald. Dit betekent dat NLP en Sentiment Analysis al enorm gevorderd zijn.
In figuur \ref{fig:modellen} staan de modellen die in de proof of concept besproken werden. Het valt op dat alle modellen een redelijk goede score behaald hebben. Echter was model vier, wat een RNN (Recurrent Neural Netwerk) was, het beste model. Dit model voldeed wel aan de vooraf gedefinieerde 80\%.

\begin{figure}[!htbp]
    \includegraphics[width=\textwidth]{Grafieken/Modellen.PNG}
    \caption{\label{fig:modellen}Resultaten van alle modellen.}
\end{figure}
\FloatBarrier

In figuur \ref{fig:methodes} staat het beste model van de proof of concept in vergelijking met de twee externe methodes. Hier valt op dat Google Cloud Platform de beste gemiddelde resultaten opleverde. 

\begin{figure}[!htbp]
    \includegraphics[width=\textwidth]{Grafieken/methodes.PNG}
    \caption{\label{fig:methodes}Resultaten van alle methodes.}
\end{figure}
\FloatBarrier

Hieruit kan geconcludeerd worden dat NLP al heel ver staat, maar dat er nog ruimte is voor verbetering.

\section{Kunnen AI's bepaalde gevoelswaarden al begrijpen?}
\label{sec:gevoelswaarden}
Ja. Het is duidelijk dat AI's gemakkelijk het verschil tussen positieve, neutrale en negatieve reviews kunnen analyseren. Door de goede resultaten op de drie datasets, kan geconcludeerd worden dat NLP en AI's goed gevoelswaarden kunnen herkennen.

\section{Kunnen bedrijven Sentiment Analysis en Opinion Mining gebruiken bij het analyseren van hun reviews?}
\label{sec:reviews}
Het is enorm handig voor bedrijven om met de klant te kunnen communiceren aan de hand van gevoelswaarden die uit hun reviews afgeleid worden.
Het vooraf gedefinieerde succespercentage van 80\% werd bereikt. Daaruit kan besloten worden dat bedrijven dit zeker in de praktijk kunnen toepassen. Dit verhoogt immers de verstandshouding tussen het bedrijf en de klant.

\section{Verder Onderzoek}
\label{sec:verderonderzoek}
Natuurlijk is verder onderzoek nog vereist. Ten eerste zou het interessant zijn om verder onderzoek te doen naar hoe Artificiële Intelligentie gepast kan reageren op een review. Nu het duidelijk is dat AI gevoelswaarden uit een review kan afleiden, is verder onderzoek naar hoe men hierop reageert belangrijk. Verder kan er ook onderzoek gedaan worden naar modellen of methodes die een succespercentage van 90\% kunnen bereiken. Dit betekent meer zekerheid voor de bedrijven dat de AI juist reageert op de wensen van de klant.



%%=============================================================================
%% Bijlagen
%%=============================================================================

\appendix
\renewcommand{\chaptername}{Appendix}

%%---------- Onderzoeksvoorstel -----------------------------------------------

\chapter{Onderzoeksvoorstel}

Het onderwerp van deze bachelorproef is gebaseerd op een onderzoeksvoorstel dat vooraf werd beoordeeld door de promotor. Dat voorstel is opgenomen in deze bijlage.

% Verwijzing naar het bestand met de inhoud van het onderzoeksvoorstel
%---------- Inleiding ---------------------------------------------------------

\section{Introductie} % The \section*{} command stops section numbering
\label{sec:introductie}

Artificiële intelligentie heeft de afgelopen jaren al heel wat vooruitgang geboekt. Ook op het vlak van vertalingen zien we veel tools opduiken zoals Google Translate en DeepL die een redelijk goede vertaling aan de gebruiker kunnen aanbieden. Vaak volstaan deze vertalingen niet omdat Artificiële Intelligentie geen emotionele gevoelswaarden of culturele aspecten, zoals beleefdheidsvormen, kan verstaan of vertalen. De betekenis of zelfs context van veel zinnen gaat zo verloren. Maar zou het niet handig zijn moest een AI gevoelswaarden kunnen afleiden uit een tekst? Denk maar aan bedrijven waarvoor de klanten recensies kunnen schrijven. De AI zou dan aan de hand van detectiesoftware bepaalde gevoelswaarden kunnen herkennen. Het bedrijf in kwestie kan dan inspelen op de gevoelens van elke afzonderlijke klant, wat enorm veel voordelen met zich oplevert.
Een voorbeeld hiervan: een klant laat een boze, ontevreden recensie achter. De AI detecteert dat dit een boze recensie is, en speelt hierop in met een verontschuldigend antwoord en misschien zelfs een compensatie.  
In deze bachelorproef zal dit thema onderzocht worden: kunnen AI’s getraind worden om gevoelswaarden in een tekst te begrijpen en hiernaar correct te handelen? Daarbij worden volgende vragen onderzocht:

\begin{itemize}
  \item Hoe ver staan Natural Language Processing en Sentiment Analysis al?
  \item Kunnen AI's bepaalde gevoelswaarden al begrijpen?
  \item In welke mate kunnen AI's deze gevoelswaarden begrijpen?
  \item Kunnen bedrijven Sentiment Analysis en Opinion Mining gebruiken bij het analyseren van hun reviews?
\end{itemize}

%---------- Stand van zaken ---------------------------------------------------

\section{State-of-the-art}
\label{sec:state-of-the-art}

Artificiële Intelligentie heeft al heel wat vorderingen gemaakt op het vlak van het analyseren van teksten. Maar deze analyses verstaan nooit de gevoelswaarde die de auteur in de tekst probeert te brengen. 
Het vakgebied van artificiële intelligentie dat hierover handelt, heet Natural Language Processing of NLP. NLP houdt zich vooral bezig met de interactie tussen computers en menselijke taal. NLP onderzoekt of een AI in staat is om de inhoud van een tekst te begrijpen. De uitdaging is hier om ook gevoelens uit deze teksten te halen.~\autocite{Knight1997}
Een vaak gebruikte techniek bij NLP is sentiment analysis, ook bekend als opinion mining. Deze techniek bepaalt of data een positieve, negatieve of neutrale connotatie heeft. ~\autocite{MonkeyLearn2020}
Er zijn al een aantal onderzoeken gedaan naar de mogelijkheid om AI’s de kennis over gevoelswaarden te geven. ~\autocite{ChewYean2015} Microsoft heeft zo een Tekst Analytics API ontworpen. ''De Text Analytics API is een cloud-gebaseerde dienst die Natural Language Processing (NLP) functies biedt voor text mining en tekstanalyse, waaronder: sentiment analysis, opinion mining,key phrase extraction, language detection en named entity recognition.'' ~\autocite{Microsoft2020} Er werd hiervoor een  Text Analytics Library ontworpen. Hierbij kan men over een zin zeggen of deze een negatieve, neutrale of positieve connotatie heeft. Dit onderzoek zal verder uitgewerkt worden in de bachelorproef.
Verder zal er geprobeerd worden om zelf een dataset te trainen en de hoogst mogelijke accuracy proberen te halen met behulp van een al eerder aangemaakte dataset van een aantal reviews ~\autocite{Minqing2004}. 


% Voor literatuurverwijzingen zijn er twee belangrijke commando's:
% \autocite{KEY} => (Auteur, jaartal) Gebruik dit als de naam van de auteur
%   geen onderdeel is van de zin.
% \textcite{KEY} => Auteur (jaartal)  Gebruik dit als de auteursnaam wel een
%   functie heeft in de zin (bv. ``Uit onderzoek door Doll & Hill (1954) bleek
%   ...'')


%---------- Methodologie ------------------------------------------------------
\section{Methodologie}
\label{sec:methodologie}

Om uit te testen hoe ver AI al staat met het analyseren van een tekst en de gevoelens van deze tekst, zal de hierboven vermelde Microsoft library getest worden in Visual Studio. Er zullen een honderdtal zinnen weergegeven worden, en op basis van deze zinnen zal er gekeken worden hoe goed deze tool is om recensies en opmerkingen te analyseren. Zinnen die goed geanalyseerd worden zullen een 1 toegekend krijgen, terwijl zinnen die slecht geanalyseerd worden een 0 krijgen. Onder 'goed' verstaan we een positieve zin die beoordeeld wordt als positief, een negatieve zin als negatief en een neutrale zin als neutraal. 
In dit onderzoek zal de data van een al bestaande dataset getraind worden om te kijken of AI al ver genoeg staat zodat bedrijven hun recensies zouden kunnen laten analyseren. Verder onderzoek over welke dataset zal gebruikt worden is nog nodig.


%---------- Verwachte resultaten ----------------------------------------------
\section{Verwachte resultaten}
\label{sec:verwachte_resultaten}

Wat er verwacht wordt van de resultaten na de testen uitgevoerd te hebben, is dat Artificiële Intelligentie al ver staat, maar dat het detecteren van emoties in teksten nog niet perfect is. Er zijn al heel wat studies en experimenten uitgevoerd over dit onderwerp, maar er zijn enorm veel nuances aan een taal, waardoor verwacht wordt dat de huidige tools de tekst nog niet perfect kunnen evalueren. Voor het evalueren van de gevoelswaarden van recensies voor bedrijven wordt er verwacht dat de huidige tools en de huidige kennis wel genoeg zijn.
Bij de Microsoft Library worden redelijk goede resultaten verwacht. Er wordt verwacht dat 85-90 procent van de zinnen juist geanalyseerd zal worden. De Microsoft Library geeft enkel aan of een zin een positieve, negatieve of neutrale connotatie heeft. Bij het trainen van een al eerder gedefineerde dataset, met eigen geschreven training en testing worden er iets minder goede resultaten verwacht. We verwachten hier dat 75-80 procent van de zinnen correct geanalyseerd zal worden.



%---------- Verwachte conclusies ----------------------------------------------
\section{Verwachte conclusies}
\label{sec:verwachte_conclusies}

De conclusie die verwacht wordt is de volgende: Artificiële Intelligentie is wel klaar om door bedrijven toegepast te worden om het gevoel van hun klanten in reviews of teksten te herkennen en hier gepast op te handelen. Het is echter heel moeilijk voor AI’s om het menselijk denken en handelen in een dataset te plaatsen en deze te analyseren. Ondanks dat AI nog niet in staat is om alle zinnen correct te analyseren, staat AI wel al ver genoeg om opinion mining in het echte leven bij bedrijven te gebruiken.



%%---------- Andere bijlagen --------------------------------------------------
\chapter{Verklarende woordenlijst}
% TODO: Voeg hier eventuele andere bijlagen toe
%%=============================================================================
%% Verklarende Woordenlijst
%%=============================================================================


Dit is een lijst van alle belangrijke termen die in deze bachelorproef worden aangehaald. Wanneer de lezer tijdens het lezen een opfrisser wil van een bepaalde term, kan hij of zij hier terecht.

\printglossary[title=Glossary, toctitle=List of terms]

    

\clearpage

%%---------- Referentielijst --------------------------------------------------

\printbibliography[heading=bibintoc]

\end{document}
