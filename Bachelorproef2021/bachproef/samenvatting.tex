%%=============================================================================
%% Samenvatting
%%=============================================================================

% TODO: De "abstract" of samenvatting is een kernachtige (~ 1 blz. voor een
% thesis) synthese van het document.
%
% Deze aspecten moeten zeker aan bod komen:
% - Context: waarom is dit werk belangrijk?
% - Nood: waarom moest dit onderzocht worden?
% - Taak: wat heb je precies gedaan?
% - Object: wat staat in dit document geschreven?
% - Resultaat: wat was het resultaat?
% - Conclusie: wat is/zijn de belangrijkste conclusie(s)?
% - Perspectief: blijven er nog vragen open die in de toekomst nog kunnen
%    onderzocht worden? Wat is een mogelijk vervolg voor jouw onderzoek?

\IfLanguageName{english}{%
\selectlanguage{dutch}
\chapter*{Samenvatting}
\lipsum[1-4]
\selectlanguage{english}
}{}

%%---------- Samenvatting -----------------------------------------------------
% De samenvatting in de hoofdtaal van het document

\chapter*{\IfLanguageName{dutch}{Samenvatting}{Abstract}}

Artificiële Intelligentie is een concept dat de laatste jaren meer en meer in opmars is. Natural Language Processing, het deelveld van AI dat zich met taal bezighoudt, kent steeds meer vooruitgang. Denk maar aan Alexa, Siri, Google Translate of spraakgestuurde huizen. Ook bedrijven kijken vaker in de richting van NLP om de reviews te kunnen analyseren. Het is immers handig om de review van een klant als positief, neutraal of negatief in te kunnen delen. Er kan dan gepast gereageerd worden op de negatieve reviews. Dit zorgt op zijn beurt voor een betere verhouding tussen bedrijf en klant.

In deze bachelorproef werd onderzocht in welke mate AI deze menselijke gevoelswaarden al kan analyseren en er wordt in de conclusie beslist of NLP klaar is om in de bedrijfswereld toegepast te worden. Ook werd er onderzocht hoe ver NLP al staat op dit vlak, en werd er zelf een proof-of-concept geschreven om te beoordelen of het implementeren van een AI vlot kan plaatsvinden. 

Om te onderzoeken hoe ver NLP al staat, werden twee externe tools beoordeeld. Een nauwkeurigheid van 80\% betekende dat de tools goed genoeg waren om in het bedrijfsleven toegepast te worden. Een eerste tool, Azure Text Analytics API scoorde heel goed op een dataset met reviews, maar minder goed op een dataset met tweets. Als enkel de reviews beoordeeld worden, zou deze tool meer dan goed genoeg zijn. Voor kortere zinnen, zoals bij tweets, scoorde deze tool ondermaats. De tweede tool, Google Cloud Platform, scoorde op alle datasets veel beter dan de Azure Text Analytics API. 

Ten slotte werd in de proof-of-concept een ander model gebruikt om de reviews en tweets te analyseren. Model 1 leverde zowel voor de reviews als voor de tweets goede resultaten op. Model 2 ...

MODEL 2 MOET IK NOG SCHRIJVEN

Men kan concluderen dat Natural Language Processing klaar is om in de bedrijfswereld gebruikt te worden. Het is het verstandigst om een tool als Google Cloud Platform te gebruiken om zinnen te analyseren. Deze leverde immers de beste resultaten op.

In de toekomst kan het interessant zijn om te onderzoeken hoe er gepast kan gereageerd worden op een negatieve review. Ten eerste zal het onderwerp van de review moeten bepaald worden. Ten tweede zal de AI dan de juiste tekst moeten formuleren om zo de klant tevreden te houden. Er is ook nog verbetering mogelijk op vlak van nauwkeurigheid. In de toekomst kan er onderzoek gedaan worden naar modellen die een nauwkeurigheid van 90\% kunnen halen.







