%%=============================================================================
%% Inleiding
%%=============================================================================

\chapter{\IfLanguageName{dutch}{Inleiding}{Introduction}}
\label{ch:inleiding}

Artificiële intelligentie heeft de afgelopen jaren al heel wat vooruitgang geboekt. Ook op het vlak van vertalingen zien we veel tools opduiken zoals Google Translate en DeepL die een redelijk goede vertaling aan de gebruiker kunnen aanbieden. Vaak volstaan deze vertalingen niet omdat Artificiële Intelligentie geen emotionele gevoelswaarden of culturele aspecten, zoals beleefdheidsvormen, kan verstaan of vertalen. De betekenis of zelfs context van veel zinnen gaat zo verloren. Maar zou het niet handig zijn moest een AI gevoelswaarden kunnen afleiden uit een tekst? Denk maar aan bedrijven waarvoor de klanten recensies kunnen schrijven. De AI zou dan aan de hand van detectiesoftware bepaalde gevoelswaarden kunnen herkennen. Het bedrijf in kwestie kan dan inspelen op de gevoelens van elke afzonderlijke klant, wat enorm veel voordelen met zich oplevert.
Een voorbeeld hiervan: een klant laat een boze, ontevreden recensie achter. De AI detecteert dat dit een boze recensie is, en speelt hierop in met een verontschuldigend antwoord en misschien zelfs een compensatie.  
In deze bachelorproef zal dit thema onderzocht worden: kunnen AI’s getraind worden om gevoelswaarden in een tekst te begrijpen en hiernaar correct te handelen? Daarbij worden volgende vragen onderzocht:

\begin{itemize}
    \item Hoe ver staan Natural Language Processing en Sentiment Analysis al?
    \item Kunnen AI's bepaalde gevoelswaarden al begrijpen?
    \item In welke mate kunnen AI's deze gevoelswaarden begrijpen?
    \item Kunnen bedrijven Sentiment Analysis en Opinion Mining gebruiken bij het analyseren van hun reviews?
\end{itemize}


\section{\IfLanguageName{dutch}{Probleemstelling}{Problem Statement}}
\label{sec:probleemstelling}

Veel bedrijven die producten of diensten aanbieden hebben een pagina waar de klant reviews kan achterlaten. Het is echter niet gemakkelijk om elke review te bekijken, analyseren en hier gepast naar te handelen. Het doel van reviews is om de klant en zijn gevoelens beter te begrijpen. Het zou een enorme meerwaarde voor het bedrijf bieden, moesten de reviews van de klant geanalyseerd kunnen worden door artificiële intelligentie. Zo kan er gepast gereageerd worden op een review of klacht. 

\section{\IfLanguageName{dutch}{Onderzoeksvraag}{Research question}}
\label{sec:onderzoeksvraag}

Om te onderzoeken of artificiële intelligentie sentimentele data kan halen uit reviews en/of klachten, luidt de onderzoeksvraag van deze bachelorproef als volgt: `In welke mate kunnen AI’s getraind worden om menselijke gevoelswaarden te herkennen in teksten?'. Deze gevoelswaarden zullen positief of negatief blijken te zijn. Ook worden er een paar deelvragen onderzocht om het onderwerp beter te kaderen:

\begin{itemize}
    \item Hoe ver staan Natural Language Processing en Sentiment Analysis al?
    \item Kunnen AI's bepaalde gevoelswaarden al begrijpen?
    \item In welke mate kunnen AI's deze gevoelswaarden begrijpen?
    \item Kunnen bedrijven Sentiment Analysis en Opinion Mining gebruiken bij het analyseren van hun reviews?
\end{itemize}

Op het einde van deze bachelorproef zal duidelijk zijn of Sentiment Analysis kan gebruikt worden door bedrijven om een review te analyseren. Verder wordt er ook onderzocht of dit effectief is. Haalt het bedrijf hier echt voordeel uit? Zijn de voorspellingen voor een review accuraat?


\section{\IfLanguageName{dutch}{Onderzoeksdoelstelling}{Research objective}}
\label{sec:onderzoeksdoelstelling}

Het resultaat dat deze bachelorproef wil bereiken is om een betere verstandshouding tussen het bedrijf en de klant te scheppen. Wanneer reviews kunnen geanalyseerd worden door een machine, kan deze machine de review indelen als positief of negatief. Daarna kan er gepast geantwoord worden op de review van de klant. 

Deze bachelorproef kan als 'succesvol' gezien worden wanneer een conclusie bereikt wordt over of artificiële intelligentie klaar is om sentiment analysis toe te passen op reviews. Een succes-percentage van 80 procent wordt gezien als succesvol. Er wordt een proof-of-concept gedaan om te kijken of bepaalde tools dit percentage kunnen bereiken. Verder wordt er ook een eigen versie geschreven om dit percentage te bereiken.

\section{\IfLanguageName{dutch}{Opzet van deze bachelorproef}{Structure of this bachelor thesis}}
\label{sec:opzet-bachelorproef}

% Het is gebruikelijk aan het einde van de inleiding een overzicht te
% geven van de opbouw van de rest van de tekst. Deze sectie bevat al een aanzet
% die je kan aanvullen/aanpassen in functie van je eigen tekst.

De rest van deze bachelorproef is als volgt opgebouwd:

In Hoofdstuk~\ref{ch:stand-van-zaken} wordt een overzicht gegeven van de stand van zaken binnen het onderzoeksdomein, op basis van een literatuurstudie.

In Hoofdstuk~\ref{ch:methodologie} wordt de methodologie toegelicht en worden de gebruikte onderzoekstechnieken besproken om een antwoord te kunnen formuleren op de onderzoeksvragen.

% TODO: Vul hier aan voor je eigen hoofstukken, één of twee zinnen per hoofdstuk

In Hoofdstuk~\ref{ch:conclusie}, tenslotte, wordt de conclusie gegeven en een antwoord geformuleerd op de onderzoeksvragen. Daarbij wordt ook een aanzet gegeven voor toekomstig onderzoek binnen dit domein.