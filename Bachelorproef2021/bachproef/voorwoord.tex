%%=============================================================================
%% Voorwoord
%%=============================================================================

\chapter*{\IfLanguageName{dutch}{Woord vooraf}{Preface}}
\label{ch:voorwoord}

Ik heb altijd al een passie gehad voor talen. Daarom ben ik in 2015 aan de opleiding Bedrijfsvertaler-tolk begonnen aan de Hogeschool Gent. Na dit diploma behaald te \\hebben, voelde ik dat er nog iets miste. Ik had namelijk ook een grote interesse in IT, webapplicaties en apps. Daarom ving ik in 2018 de studierichting Toegepaste Informatica aan, eveneens aan de Hogeschool Gent.

Om mijn twee grote passies te combineren, kwam ik al snel met het idee op de proppen om met NLP en AI te werken. Het is immers heel interessant om te zien hoe twee volledig verschillende richtingen toch samen kunnen komen op bepaalde raakvlakken.

Ik kwam het eerst in aanraking met Artificiële Intelligentie in het derde jaar tijdens de lessen Artificiële Intelligentie en Databanken III. Ik vond dit direct een intrigerend \\onderwerp en heb daarop ook mijn keuze voor het onderwerp van deze bachelorproef gebaseerd. 

Deze bachelorproef is de kers op de taart van mijn laatste jaar Toegepaste Informatica en ben dan ook enorm trots om dit te hebben verwezenlijkt. Maar ik zou dit zeker niet alleen gekunnen hebben.

Allereerst wil ik mijn promotor, Leen Vuyge bedanken om me bij te staan met goede raad en om steeds mijn taalfoutjes uit de bachelorproef te halen. Het was een enorm voorrecht om met u samen te werken en samen dit concept uit te denken.

Ook wil ik mijn twee co-promotors Robin Menschaert en Lothar Verledens bedanken voor de expertise en suggesties. Dankzij hun durfde ik het aan om ook zelf een proof of concept te proberen schrijven.

Ten slotte wil ik mijn revisoren bedanken. Bedankt Joppe Eggermont, Joachim 'Jote' Demyttenaere, Isabelle Schelstraete en Lieven Moerman om mijn bachelorproef na te lezen en alle foutjes eruit te halen.

Verder wens ik u veel leesplezier toe en hoop ik dat u een beter inzicht krijgt in NLP en Opinion Mining.

Eline Moerman \\
28/05/2021, Gent
